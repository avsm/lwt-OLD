\documentclass{article}

%% +-----------------------------------------------------------------+
%% | Packages                                                        |
%% +-----------------------------------------------------------------+

\usepackage[utf8]{inputenc}
\usepackage{xspace}
\usepackage{verbatim}
\usepackage[english]{babel}
\usepackage{amsmath}
\usepackage{amssymb}

%% +-----------------------------------------------------------------+
%% | Aliases                                                         |
%% +-----------------------------------------------------------------+

\newcommand{\lwt}{\texttt{Lwt}\xspace}

%% +-----------------------------------------------------------------+
%% | Headers                                                         |
%% +-----------------------------------------------------------------+

\title{Lwt user manual}
\author{Jérémie Dimino}

\begin{document}

\maketitle

\textbf{Note:} this document is a work in progress. Sections 1 and 2
are mostly ready.

%% +-----------------------------------------------------------------+
%% | Table of contents                                               |
%% +-----------------------------------------------------------------+

\setcounter{tocdepth}{3}
\tableofcontents

%% +-----------------------------------------------------------------+
%% | Section                                                         |
%% +-----------------------------------------------------------------+
\section{Introduction}

When writing a program, a common task of the developer is to handle IO
operations. Indeed most software interact with several different
resources, such as:

\begin{itemize}
\item the kernel, by doing system calls
\item the user, by reading the keyboard, the mouse, or any input
  device
\item a graphical server, to build graphical user interface
\item other computer, by using gthe network
\item \dots
\end{itemize}

When this list contains only one item, it is pretty easy to
handle. However as this list grows it becomes harder and harder to
make everything works together. Several choices have been proposed to
solve this problem:

\begin{itemize}
\item using a main loop, and integrate all components we are
  ineteracting with into this main loop.
\item using preemptive system threads
\end{itemize}

Both solution have their advantages and their drawbacks. For the first
one, it may works, but it become very complicated to write a bit of
sequential code. The typical example being with graphical user
interfaces freezing and not redrawing themselves because they are
waiting for some blocking part of the code to complete.

If you already wrote code using preemptive threads, you shall know
that doing it right with threads is a hard job. Moreover system
threads consume non negligible resources, and so you are limited in
the number of threads you launch at the same time. Thus this is not a
real solution.

\lwt offers a new alternative. It provides very light-weight
cooperative threads; ``launching'' a thread is a very quick operation,
it does not require a new stack, a new process, or anything
else. Moreover context switches are very fast. In fact, it is so easy
that we will launch a thread for every system call. And composing
cooperative threads will allow us to write highly asynchronous
programs.

In a first part, we will explain the concepts of \lwt, then we will
describe the many sub-libraries of \lwt.

%% +-----------------------------------------------------------------+
%% | Section                                                         |
%% +-----------------------------------------------------------------+
\section{The Lwt core library}

In this section we describe the basis of \lwt. It is advised to start
an ocaml toplevel and try codes given as example. To start, launch
\texttt{ocaml} in a terminal or in emacs with the tuareg mode, and
type:

\begin{verbatim}
# #use "topfind";;
# #require "lwt";;
\end{verbatim}

Also \lwt is now chipped with an improved toplevel, which support line
edition and completion. If it has been correctly installed, you should
be able to start it with the following command:

\begin{verbatim}
$ ocamlfind lwt/toplevel.top
\end{verbatim}

\subsection{Lwt concepts}

Let's take a classical function of the \texttt{Pervasives} module:

\begin{verbatim}
# Pervasives.input_char;
- : in_channel -> char = <fun>
\end{verbatim}

This function will wait for a character to come on the given input
channel, then return it. The problem with this function is that it is
blocking: while it is being executed, the whole program will be
blocked, and other events will not be handled until it returns.

Now let's look what is the lwt equivalent:

\begin{verbatim}
# Lwt_io.read_char;;
- : Lwt_io.input_channel -> char Lwt.t = <fun>
\end{verbatim}

As you can see, it does not returns a character but something of type
\texttt{char Lwt.t}. The type \texttt{'a Lwt.t} is the type of threads
returning a value of type \texttt{'a}. Actually the
\texttt{Lwt\_io.read\_char} will try to read a character to the given
input channel and \emph{immediatly} returns a light-weight thread.

Now, let's see what we can do a \lwt thread. The following code create
a pipe, and launch a thread reading on the input side:

\begin{verbatim}
# let ic, oc = Lwt_io.pipe ();;
val ic : Lwt_io.input_channel = <abstr>
val oc : Lwt_io.output_channel = <abstr>
# let t = Lwt_io.read_char ic;;
val t : char Lwt.t = <abstr>
\end{verbatim}

We now introduce the concept of thread state:

\begin{verbatim}
# Lwt.state t;;
- : char Lwt.state = Sleep
\end{verbatim}

A thread may be in one of the following states:

\begin{itemize}
\item \texttt{Return x}, which means that the thread has terminated
  successfully and returned the value \texttt{x}
\item \texttt{Fail exn}, which means that the thread has terminated,
  but instead of returning a value, it failed with the exception
  \texttt{exn}
\item \texttt{Sleep}, which means that the thread is currently
  sleeping and have not yet returned a value or an exception
\end{itemize}

The thread \texttt{t} is sleeping because there is currently nothing
to read on the pipe. Let's write something:

\begin{verbatim}
# Lwt_io.write_char oc 'a';;
- : unit Lwt.t = <abstr>
# Lwt.state t;;
- : char Lwt.state = Return 'a'
\end{verbatim}

So, after we write something, the reading thread has been wakeup and
has returned the value \texttt{'a'}.

\subsubsection{Primitives for thread creation}

We now look at the primitives for creating \lwt threads. These
functions are localted in the module \texttt{Lwt}.

Here are the main primitives:

\begin{itemize}
\item ``\texttt{Lwt.return : 'a -> 'a Lwt.t}'' creates a thread which has already
  terminated and returned a value
\item ``\texttt{Lwt.fail : exn -> 'a Lwt.t}'' creates a thread which has
  already terminated and failed with an exception
\item ``\texttt{Lwt.wait : unit -> 'a Lwt.t * 'a Lwt.u}'' creates a
  sleeping thread and returns this thread plus a wakener (of type
  \texttt{'a Lwt.u}) which must be used to wakeup the sleeping thread.
\end{itemize}

To wakeup a sleeping thread, you can use one of the following
functions:

\begin{itemize}
\item ``\texttt{Lwt.wakeup : 'a Lwt.u -> 'a -> unit}'' wakeups the
  thread with a value.
\item ``\texttt{Lwt.wakeup\_exn : 'a Lwt.u -> exn -> unit}'' wakeups
  the thread with an exception.
\end{itemize}

Not that this is an error to wakeup two times the same threads. \lwt
will raise \texttt{Invalid\_argument} if you try to do so.

With these informations, try to guess the result of each of the
following expression:

\begin{verbatim}
# Lwt.state (Lwt.return 42);;
# Lwt.state (Lwt.fail Exit);;
# let waiter, wakener = Lwt.wait ();;
# Lwt.state waiter;;
# Lwt.wakeup wakener 42;;
# Lwt.state waiter;;
# let waiter, wakener = Lwt.wait ();;
# Lwt.state waiter;;
# Lwt.wakeup_exn wakener Exit;;
# Lwt.state waiter;;
\end{verbatim}

\subsubsection{Primitives for thread composition}

The most important operation you need to know is \texttt{bind}:

\begin{verbatim}
val bind : 'a Lwt.t -> ('a -> 'b Lwt.t) -> 'b Lwt.t
\end{verbatim}

\texttt{bind t f} creates a thread which waits for \texttt{t} to
terminates, then pass the result to \texttt{f}. It \texttt{t} is a
sleeping thread, then \texttt{bind t f} will be a sleeping thread too,
until \texttt{t} terminates. If \texttt{t} fails, then the resulting
thread will fail with the same exception.

Similarly to \texttt{bind}, there is a function to handle the case
when \texttt{t} fails:

\begin{verbatim}
val catch : (unit -> 'a Lwt.t) -> (exn -> 'a Lwt.t) -> 'a Lwt.t
\end{verbatim}

\texttt{catch f g} will call \texttt{f ()}, then waits for its
termination, and if it fails with an exception \texttt{exn}, calls
\texttt{g exn} to handle it. Note that both exception raised with
\texttt{Pervasives.raise} and \texttt{Lwt.fail} are caught by
\texttt{catch}.

\subsubsection{Cancelable threads}

In some case, we may want to cancel a threads. For example, because it
has not terminated after a timeout. This can be done with cancelable
threads. To create a cancelable thread, you must use the
\texttt{Lwt.task} function:

\begin{verbatim}
val task : unit -> 'a Lwt.t * 'a Lwt.u
\end{verbatim}

It has the same semantic that \texttt{Lwt.wait} except that the
sleeping thread can be canceled with \texttt{Lwt.cancel}:

\begin{verbatim}
val cancel : 'a Lwt.t -> unit
\end{verbatim}

The threads will then fails with the exception
\texttt{Lwt.Canceled}. To execute a function when the thread is
canceled, you must use \texttt{Lwt.on\_cancel}:

\begin{verbatim}
val on_cancel : 'a Lwt.t -> (unit -> unit) -> unit
\end{verbatim}

Note that it is also possible to cancel a thread which has not been
created with \texttt{Lwt.task}. In this case, the deepest cancelable
threads connected the given thread will be cancelled.

For example, consider the following code:

\begin{verbatim}
# let waiter, wakener = Lwt.task ();;
val waiter : '_a Lwt.t = <abstr>
val wakener : '_a Lwt.u = <abstr>
# let t = bind waiter (fun x -> return (x + 1));;
val t : int Lwt.t = <abstr>
\end{verbatim}

Here, cancelling \texttt{t} will in fact cancel \texttt{waiter},
\texttt{t} will then fails with the exception \texttt{Lwt.Canceled}:

\begin{verbatim}
# Lwt.cancel t;;
- : unit = ()
# Lwt.state waiter;;
- : int Lwt.state = Fail Lwt.Canceled
# Lwt.state t;;
- : int Lwt.state = Fail Lwt.Canceled
\end{verbatim}

By the way, it is possible to prevent a thread to be canceled, by
using the function \texttt{Lwt.protected}:

\begin{verbatim}
val protected : 'a Lwt.t -> 'a Lwt.t
\end{verbatim}

Canceling \texttt{proctected t} will have no effect.

\subsubsection{Primitives for multi-thread composition}

We conculde this section with multi-thread composition function. There
are three such functions in the \lwt module: \texttt{join},
\texttt{choose} and \texttt{select}.

The first one, \texttt{join} takes a list of threads, and wait for all
of them to terminates:

\begin{verbatim}
val join : unit Lwt.t list -> unit Lwt.t
\end{verbatim}

Moreover, if one thread fails, it fails with the same exception,
without waiting for the other threads to terminates.

On the contrary \texttt{choose} waits for at least one threads to
terminates, then returns the same value or exception:

\begin{verbatim}
val choose : 'a Lwt.t list -> 'a Lwt.t
\end{verbatim}

For example:

\begin{verbatim}
# let waiter1, wakener1 = Lwt.wait ();;
val waiter1 : '_a Lwt.t = <abstr>
val wakener1 : '_a Lwt.u = <abstr>
# let waiter2, wakener2 = Lwt.wait ();;
val waiter2 : '_a Lwt.t = <abstr>
val wakeneré : '_a Lwt.u = <abstr>
# let t = Lwt.choose [waiter1; waiter2];;
val t : '_a Lwt.t = <abstr>
# Lwt.state t;;
- : '_a Lwt.state = Sleep
# Lwt.wakeup wakener2 42;;
- : unit = ()
# Lwt.state t;;
- : int Lwt.state = Return 42
\end{verbatim}

\subsubsection{Rules}

\lwt will always try to execute the more it can before yielding and
switch to another cooperative threads. In order to make it works well,
you must follow the following rules:

\begin{itemize}
\item do not write function that may takes time to complete without
  using \lwt,
\item do not do IOs that may block, otherwise the whole program will
  hang. You must instead use asynchronous IOs operations.
\end{itemize}

\subsection{The syntax extension}

\lwt offers a syntax extension to make code using it more readable.
To use it add the ``lwt.syntax'' when compiling:

\begin{verbatim}
$ ocamlfind ocamlc -syntax camlp4o -package lwt.syntax -linkpkg -o foo foo.ml
\end{verbatim}

Or in the toplevel (after loading topfind):

\begin{verbatim}
# #camlp4o;;
# #require "lwt.syntax";;
\end{verbatim}

The following construction are added to the language:

\begin{itemize}
\item \texttt{lwt} $pattern_1$ \texttt{=} $expr_1$ [ \texttt{and}
  $pattern_2$ \texttt{=} $expr_2$ \dots ] \texttt{in} $expr$, which is a
  parallel let-binding construction. For example in the following
  code:

\begin{verbatim}
lwt x = f () and y = g () in
expr
\end{verbatim}

  the thread \texttt{f ()} and \texttt{g ()} are launched in parallel
  and their result are then bound to \texttt{x} and \texttt{y} in the
  expression $expr$.

  Of course you can also launch the two threads sequentially by
  writing your code like that:

\begin{verbatim}
lwt x = f () in
lwt y = g () in
expr
\end{verbatim}

\item \texttt{try\_lwt} $expr$ [ \texttt{with} $pattern_1$
  \texttt{$\rightarrow$} $expr_1$ ... ] [ \texttt{finally} $expr'$ ],
  which is the equivalent of the standard \texttt{try-with}
  construction but for \lwt. Both exception raised by
  \texttt{Pervasives.raise} and \texttt{Lwt.fail} are caught.

\item \texttt{for\_lwt} $ident$ \texttt{=} $expr_{init}$ ( \texttt{to}
  $\mid$ \texttt{downto} ) $expr_{final}$ \texttt{do} $expr$
  \texttt{done}, which is the equivalent of the standard \texttt{for}
  construction but for \lwt.
\end{itemize}

\subsubsection{Correspondence table}

You can keep in mind the following table to write code using lwt:

\begin{center}
  \begin{tabular}{|l|l|}
    \hline

    \textbf{without \lwt} & \textbf{with \lwt} \\

    \hline

    \texttt{let} $pattern_1$ \texttt{=} $expr_1$ &
    \texttt{lwt} $pattern_1$ \texttt{=} $expr_1$ \\
    \texttt{and} $pattern_2$ \texttt{=} $expr_2$ &
    \texttt{and} $pattern_2$ \texttt{=} $expr_2$ \\
    \dots &
    \dots \\
    \texttt{and} $pattern_n$ \texttt{=} $expr_n$ \texttt{in} &
    \texttt{and} $pattern_n$ \texttt{=} $expr_n$ \texttt{in} \\
    $expr$ &
    $expr$ \\

    \hline

    \texttt{try} &
    \texttt{try\_lwt} \\
    \:\: $expr$ &
    \:\: $expr$ \\
    \texttt{with} &
    \texttt{with} \\
    \:\: $\mid pattern_1$ \texttt{$\rightarrow$} $expr_1$ &
    \:\: $\mid pattern_1$ \texttt{$\rightarrow$} $expr_1$ \\
    \:\: $\mid pattern_2$ \texttt{$\rightarrow$} $expr_2$ &
    \:\: $\mid pattern_2$ \texttt{$\rightarrow$} $expr_2$ \\
    \:\: \dots &
    \:\: \dots \\
    \:\: $\mid$ $pattern_n$ \texttt{$\rightarrow$} $expr_n$ &
    \:\: $\mid$ $pattern_n$ \texttt{$\rightarrow$} $expr_n$ \\

    \hline

    \texttt{for} $ident$ \texttt{=} $expr_{init}$ \texttt{to} $expr_{final}$ \texttt{do} &
    \texttt{for\_lwt} $ident$ \texttt{=} $expr_{init}$ \texttt{to} $expr_{final}$ \texttt{do} \\
    \:\: $expr$ &
    \:\: $expr$ \\
    \texttt{done} &
    \texttt{done} \\

    \hline

    \texttt{for} $ident$ \texttt{=} $expr_{init}$ \texttt{downto} $expr_{final}$ \texttt{do} &
    \texttt{for\_lwt} $ident$ \texttt{=} $expr_{init}$ \texttt{downto} $expr_{final}$ \texttt{do} \\
    \:\: $expr$ &
    \:\: $expr$ \\
    \texttt{done} &
    \texttt{done} \\

    \hline
  \end{tabular}
\end{center}

\subsection{Other modules of the core library}

The core library contains several modules that depends only on
\lwt. The following naming convenetion is used in \lwt: when a
function takes as argument a function returning a thread that is going
to be executed sequentially, it is suffixed with ``\texttt{\_s}''. And
when it is going to be executed in parallel, it is suffixed with
``\texttt{\_p}''. For example, in the \texttt{Lwt\_list} module we have:

\begin{verbatim}
val map_s : ('a -> 'b Lwt.t) -> 'a list -> 'b list Lwt.t
val map_p : ('a -> 'b Lwt.t) -> 'a list -> 'b list Lwt.t
\end{verbatim}

\subsubsection{Mutexes}

\texttt{Lwt\_mutex} provides mutexes for \lwt. It use is almost the
same than the \texttt{Mutex} module of the thread library shipped with
OCaml. In general, programs using \lwt do not need a lot of
mutexes. They are only usefull for serialising operations.

\subsubsection{Lists}

The \texttt{Lwt\_list} module defines iteration and scanning functions
over list, similar to the ones of the \texttt{List} module, but using
functions that return a thread. For example:

\begin{verbatim}
val iter_s : ('a -> unit Lwt.t) -> 'a list -> unit Lwt.t
val iter_p : ('a -> unit Lwt.t) -> 'a list -> unit Lwt.t
\end{verbatim}

In \texttt{iter\_s f l}, \texttt{iter\_s} will can f on each elements
of \texttt{l}, waiting for completion between each elements. On the
contrary, in \texttt{iter\_p f l}, \texttt{iter\_p} will call f on all
elements of \texttt{l}, then wait for all the threads to terminate.

\subsubsection{Data streams}

\lwt streams are used in a lot of places in \lwt and its sub
libraries. They offer a high-level interface to manipulate data flows.

A stream is an object which returns elements sequentially and
lazily. Lazily means that the source of the stream is guessed for new
elements only when needed. This module contains a lot of stream
transformation, iteration, and scanning functions.

The common way of creating stream is by using
\texttt{Lwt\_stream.from} or by using
\texttt{Lwt\_stream.push\_stream}:

\begin{verbatim}
val from : (unit -> 'a option Lwt.t) -> 'a Lwt_stream.t
val push_stream : unit ->
   ([ `Data of 'a
    | `End_of_stream
    | `Exn of exn ] -> unit) * 'a Lwt_stream.t
\end{verbatim}

As for streams of the standard library, \texttt{from} takes as
argument a function which is used to create new elements.

\texttt{push\_stream} returns a function used to push new elements
into the stream and the stream which will receive them.

For example:

\begin{verbatim}
# let push, stream = Lwt_stream.push_stream ();;
val push : [ `Data of '_a | `End_of_stream | `Exn of exn ] -> unit = <fun>
val stream : '_a Lwt_stream.t = <abstr>
# push (`Data 1);;
- : unit = ()
# push (`Data 2);;
- : unit = ()
# push (`Data 3);;
- : unit = ()
# Lwt.state (Lwt_stream.next stream);;
- : int Lwt.state = Return 1
# Lwt.state (Lwt_stream.next stream);;
- : int Lwt.state = Return 2
# Lwt.state (Lwt_stream.next stream);;
- : int Lwt.state = Return 3
# Lwt.state (Lwt_stream.next stream);;
- : int Lwt.state = Sleep
\end{verbatim}

Note that streams are consumable. Once you take an element from a
stream, it is removed from it. So, if you want to iterates two times
over a stream, you may consider ``clonning'' it, with
\texttt{Lwt\_stream.clone}. Cloned stream will returns the same
elements in the same order. Consuming one will not consume the other.
For example:

\begin{verbatim}
# let s = Lwt_stream.of_list [1; 2];;
val s : int Lwt_stream.t = <abstr>
# let s' = Lwt_stream.clone s;;
val s' : int Lwt_stream.t = <abstr>
# Lwt.state (Lwt_stream.next s);;
- : int Lwt.state = Return 1
# Lwt.state (Lwt_stream.next s);;
- : int Lwt.state = Return 2
# Lwt.state (Lwt_stream.next s');;
- : int Lwt.state = Return 1
# Lwt.state (Lwt_stream.next s');;
- : int Lwt.state = Return 2
\end{verbatim}

\subsubsection{Mailbox variables}

The \texttt{Lwt\_mvar} module provides mailbox variables. A mailbox
variable, also called a ``mvar'', is a cell with may contains 0 or 1
elements. If it contains no elements, we say that the mvar is empty,
if it contains one, we say that it is full. Adding an element to a
full mvar will block until the one is taken. Taking an element from an
empty mvar will block until one is added.

Mailbox variables are commonly used to pass messages between threads.

Note that a mailbox variable can be seen as aa pushable stream with a
limited memory.

\subsubsection{Reactive programming helpers}

The \texttt{Lwt\_event} and \texttt{Lwt\_signal} provides helpers for
using the \texttt{react} library with \lwt. Among them we have
\texttt{Lwt\_event.next}, which takes an event and returns a thread
which will wait until the next occurence of this event. For example:

\begin{verbatim}
# let event, push = React.E.create ();;
val event : '_a React.event = <abstr>
val push : '_a -> unit = <fun>
# let t = Lwt_event.next event;;
val t : '_a Lwt.t = <abstr>
# Lwt.state t;;
- : '_a Lwt.state = Sleep
# push 42;;
- : unit = ()
# Lwt.state t;;
- : int Lwt.state = Return 42
\end{verbatim}

Another interesting features is the ability to limit events
(resp. signals) to occurs (resp. to changes) too often. For example,
suppose you are doing a program which display something on the screeen
each time a signal change. If at some point the signal changes 1000
times per second, you probably want not to render it 1000 times per
second. For that you use \texttt{Lwt\_signal.limit}:

\begin{verbatim}
val limit : (unit -> unit Lwt.t) -> 'a React.signal -> 'a React.signal
\end{verbatim}

\texttt{Lwt\_signal.limit f signal} returns a signal with vary as
\texttt{signal} except that two consecutive updates are separeted by a
call to \texttt{f}. For example if \texttt{f} returns a thread which
sleep for 0.1 seconds, then there will be no more than 10 changes per
second. For example:

\begin{verbatim}
let draw x =
  (* Draw the screen *)
  ...

let () =
  (* The signal we are interested in: *)
  let signal = ... in

  (* The limited signal: *)
  let signal' = Lwt_signal.limit (fun () -> Lwt_unix.sleep 0.1) signal in

  (* Redraw the screen each time the limited signal change: *)
  Lwt_signal.always_notify_p draw signal'
\end{verbatim}

%% +-----------------------------------------------------------------+
%% | Section                                                         |
%% +-----------------------------------------------------------------+
\section{The lwt.unix library}
\label{lwt.unix}

The package \texttt{lwt.unix} contains all \texttt{unix} dependant
modules of \lwt. Among all its features, it implements cooperative
versions of functions of the standard library and unix library.

\subsection{Unix primitives}

The \texttt{Lwt\_unix} provides cooperative system calls. For example,
the \lwt counterpart of \texttt{Unix.read} is:

\begin{verbatim}
val read  : file_descr -> string -> int -> int -> int Lwt.t
\end{verbatim}

\texttt{Lwt\_io} provides similar features that buffered channels of
the standard library (of type \texttt{in\_channel} or
\texttt{out\_channel}) but cooperatively.

\texttt{Lwt\_gc} allow you to register finaliser that return a
thread. At the end of the program, \lwt will wait for all the
finaliser to terminates.

\subsection{The lwt scheduler}

The \texttt{Lwt\_main} contains the \lwt \emph{main loop}. It can be
customized by adding filters, and/or by replacing the \texttt{select}
function.

Filters are responsible to collect sources to monitor before entering
the blocking \texttt{select}, then to react and wakeup threads waiting
for sources to become ready.

\subsection{Limitations}

To avoid blocking, \lwt puts all file descriptors (except
\texttt{stdin}, \texttt{stdout} and \texttt{stderr}) in non-blocking
mode. Unfortunatly no unix supports non-blocking mode for regular
files. In practise a file descriptor for a regular file will always
appears to be readable and writable even if this is not the case (for
example because the file is located on a network file system and the
server is not available). There is no workaround for this.

By the way, since reading/writing a file will never appears to block,
reading or writing a big file may prevent other threads to run until
the operation on the file are finished. To avoid this,
\texttt{Lwt\_io} yield every 0.05 seconds for regular files.

Since there is no support at all for asynchrnously reading the
contents of a directory, \texttt{Lwt\_directory} uses the same trick.

\subsection{The logging facility}
\label{lwt.log}

The package \texttt{lwt.unix} contains a module \texttt{Lwt\_log}
providing loggers. It support logging to a file, a channel, or to the
syslog daemon. You can also defines your own logger by providing the
appropriate functions (function \texttt{Lwt\_log.make}).

Several loggers can be merged into one. Sending logs on the merged
loggers will send these logs to all its components.

For example to redirect all logs to \texttt{stderr} and to the syslog
daemon:

\begin{verbatim}
# Lwt_log.default_logger :=
    Lwt_log.merge ~level:Lwt_log.Info [
      Lwt_log.channel
        ~level:Lwt_log.Info
        ~close_mode:`keep
        ~channel:Lwt_io.stderr ();
      Lwt_log.syslog
        ~level:Lwt_log.Info
        ~facility:`User ();
    ]
;;
\end{verbatim}

\lwt also provides a syntax extension, in the package
\texttt{lwt.syntax.log}. It adds the following construction to the
language:

\begin{center}
  \texttt{Log\#}\textit{level} \textit{format} \textit{arguments}
\end{center}

Where \textit{level} is one of \texttt{debug}, \texttt{info},
\texttt{warning}, \texttt{error}, \texttt{fatal}. For example:

\begin{verbatim}
Log#debug "x = %s" x
\end{verbatim}

The advantages of using the syntax extension are the following:

\begin{itemize}
\item it automatically prefix the format with the module name
\item it check the log level before calling the logging function, so
  arguments are not computed if not needed
\item debugging logs can be removed at parsing time
\end{itemize}

By default, the syntax extension remove all logs with the level
\texttt{debug}. To keep them pass the command line option
\texttt{-lwt-debug} to camlp4.

\subsection{Detaching computation to preemptive threads}

It may happen that you want to run a function which will take time to
compute or that you want to use a blocking function that cannot be
used in a non-blocking way. For these situations, \lwt allow you to
\emph{detach} the computation to a preemptive thread.

This is done by the module \texttt{Lwt\_preemptive} of the
\texttt{lwt.preemptive} package which maintains a spool of system
threads. The main function is:

\begin{verbatim}
val detach : ('a -> 'b) -> 'a -> 'b Lwt.t
\end{verbatim}

\texttt{detach f x} will execute \texttt{f x} in another thread and
asynchronously wait for the result.

The \texttt{lwt.extra} package provides wrappers for a few blocking
functions of the standard C library like \texttt{gethostbyname} (in
the module \texttt{Lwt\_lib}).

%% +-----------------------------------------------------------------+
%% | Section                                                         |
%% +-----------------------------------------------------------------+
\section{The lwt.text library}

The \texttt{lwt.text} library provides functions to deal with text
mode (in a terminal). It is composed of three following modules:

\begin{itemize}
\item \texttt{Lwt\_text}, which is the equivalent of \texttt{Lwt\_io}
  but for unicode text channels
\item \texttt{Lwt\_term}, providing various terminal utilities, such as
  reading a key from the terminal
\item \texttt{Lwt\_read\_line}, which provides functions to input text
  from the user with line editing support
\end{itemize}

\subsection{Text channels}

A text channel is basically a byte channel plus an encoding. Input
(resp. output) text channels decode (resp. encode) unicode characters
on the fly. By default, output text channels use transliteration, so
they will not fails because text you want to print cannot be encoded
in the system encoding.

For example, with you locale sets to ``C'', and the variable
\texttt{name} set to ``Jérémie'', you got:

\begin{verbatim}
# lwt () = Lwt_text.printlf "My name is %s" name;;
My name is J?r?mie
\end{verbatim}

\subsection{Terminal utilities}

The \texttt{Lwt\_term} allow you to put the terminal in \emph{raw
  mode}, meanings that input is not buffered and character are
returned as the user type them. For example, you can read a key with:

\begin{verbatim}
# lwt key = Lwt_term.read_key ();;
val key : Lwt_term.key = Lwt_term.Key_control 'j'
\end{verbatim}

The second main feature of \texttt{Lwt\_term} is the ability to prints
text with styles. For example, to print text in bold and blue:

\begin{verbatim}
# open Lwt_term;;
# lwt () = printlc [fg blue; bold; text "foo"];;
foo
\end{verbatim}

If the output is not a terminal, then \texttt{printlc} will drop
styles, and act as \texttt{Lwt\_text.printl}.

\subsection{Read-line}

\texttt{Lwt\_read\_line} provides a full featured and fully
customisable read-line implementation. You can either use the
high-level and easy to use \texttt{read\_*} functions, or use the
advanced \texttt{Lwt\_read\_line.Control.read\_*} functions.

For example:

\begin{verbatim}
# open Lwt_term;;
# lwt l = Lwt_read_line.read_line ~prompt:[text "foo> "] ();;
foo> Hello, world!
val l : Text.t = "Hello, world!"
\end{verbatim}

The second class of functions is a bit more complicated to use, but
allow to control a running read-line instances. For example you can
temporary hide it to draw something, you can send it commands, fake
input, and the prompt is a signal so it can change dynamically.

%% +-----------------------------------------------------------------+
%% | Section                                                         |
%% +-----------------------------------------------------------------+
\section{Other libraries}

\subsection{SSL support}

The package \texttt{lwt.ssl} provides the module \texttt{Lwt\_ssl}
which allow to use SSL asynchronously

\subsection{Glib integration}

The \texttt{lwt.glib} provides an alternative main loop for \lwt which
uses the glib one. This allow you to write GTK application using \lwt.
The one thing you have to do is to call \texttt{Lwt\_glib.init} at the
beginning of you program.

\end{document}
