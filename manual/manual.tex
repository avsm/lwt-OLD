\documentclass{article}

%% +-----------------------------------------------------------------+
%% | Packages                                                        |
%% +-----------------------------------------------------------------+

\usepackage[utf8]{inputenc}
\usepackage{xspace}
\usepackage{verbatim}
\usepackage[english]{babel}
\usepackage{amsmath}
\usepackage{amssymb}

%% +-----------------------------------------------------------------+
%% | Aliases                                                         |
%% +-----------------------------------------------------------------+

\newcommand{\lwt}{\texttt{Lwt}\xspace}

%% +-----------------------------------------------------------------+
%% | Headers                                                         |
%% +-----------------------------------------------------------------+

\title{Lwt user manual}
\author{Jérémie Dimino}

\begin{document}

\maketitle

%% +-----------------------------------------------------------------+
%% | Table of contents                                               |
%% +-----------------------------------------------------------------+

\setcounter{tocdepth}{2}
\tableofcontents

%% +-----------------------------------------------------------------+
%% | Section                                                         |
%% +-----------------------------------------------------------------+
\section{Introduction}

In any program, we have to interact with other processes, users, ...:

\begin{itemize}
\item the kernel, by doing system calls
\item the user, by reading the keyboard, the mouse, or any input
  device
\item a graphical server, to build graphical user interface
\item other computer, by using gthe network
\item \dots
\end{itemize}

When this list contains only one item, it is pretty easy to
handle. However as this list grow it become harder to handle. Several
choices have been proposed to solve the problem:

\begin{itemize}
\item using a main loop, and integrate all components we are
  ineteracting with into this main loop.
\item using preemptive system threads
\end{itemize}

Both solution have their advantages and their drawbacks. For the first
one, it may works, but it become very complicated to write a bit of
sequential code. The typical example being with graphical user
interfaces freezing and not redrawing themselves because they are
waiting for some blocking part of the code to complete.

If you already wrote code using preemptive threads, you shall know
that doing it right with threads is a hard job. Moreover system
threads consume non negligible resources, and so you are limited in
the number of threads you launch at the same time. So this is not a
real solution.

\lwt offers a new alternative. It provides very light-weight
cooperative threads; ``launching'' a thread is a very quick operation,
it does not require a new stack, a new process, or whatever else and
context switches are very fast. In fact, it is so easy that we will
launch a thread for every system call. And composing cooperative
threads will allow us to write higly asynchronous programs.

In a first part, we will explain the concepts of \lwt, then we will
describe the many sub-libraries of \lwt.

%% +-----------------------------------------------------------------+
%% | Section                                                         |
%% +-----------------------------------------------------------------+
\section{The Lwt core library}

In this section we describe the basis of \lwt. It is advised to start
an ocaml toplevel and try codes given as example. To start, launch
\texttt{ocaml} in a terminal or in emacs with the tuareg mode, and
type:

\begin{verbatim}
# #use "topfind";;
# #require "lwt";;
\end{verbatim}

\subsection{Lwt concepts}

Let's take a classical function of the \texttt{Pervasives} module:

\begin{verbatim}
# Pervasives.input_char;
- : in_channel -> char = <fun>
\end{verbatim}

This function will wait for a character to come on the given input
channel and return it. The problem with this function is that it is
blocking: while it is being executed, the whole program will be
blocked, and other events will not be handled until it returns.

Now let's look what is the lwt equivalent:

\begin{verbatim}
# Lwt_io.read_char;;
- : Lwt_io.input_channel -> char Lwt.t = <fun>
\end{verbatim}

As you can see, it does not returns a character but something of type
\texttt{char Lwt.t}. The type \texttt{'a Lwt.t} is the type of threads
returning a value of type \texttt{'a}. Actually the
\texttt{Lwt\_io.read\_char} will try to read a character to the given
input channel and \emph{immediatly} returns a light-weight thread.

Now, let's see what we can do a lwt thread. The following code create
a pipe, and launch a thread reading on the input side:

\begin{verbatim}
# let ic, oc = Lwt_io.pipe ();;
val ic : Lwt_io.input_channel = <abstr>
val oc : Lwt_io.output_channel = <abstr>
# let t = Lwt_io.read_char ic;;
val t : char Lwt.t = <abstr>
\end{verbatim}

We now introduce the concept of thread state:

\begin{verbatim}
# Lwt.state t;;
- : char Lwt.state = Sleep
\end{verbatim}

A thread may be in one of the following states:

\begin{itemize}
\item \texttt{Return x}, which means that the thread has terminated
  successfully and returned the value \texttt{x}
\item \texttt{Fail exn}, which means that the thread has terminated,
  but instead of returning a value, it failed with the exception
  \texttt{exn}
\item \texttt{Sleep}, which means that the thread is currently
  sleeping and have not yet returned a value or an exception
\end{itemize}

The thread \texttt{t} is sleeping because there is currently nothing
to read on the pipe. Let's write something:

\begin{verbatim}
# Lwt_io.write_char oc 'a';;
- : unit Lwt.t = <abstr>
# Lwt.state t;;
- : char Lwt.state = Return 'a'
\end{verbatim}

So, after we write something, the reading thread has been wakeup and
has returned the value \texttt{'a'}.

\subsubsection{Primitives for thread creation}

We now look at the primitives for creating \lwt threads. These
functions are localted in the module \texttt{Lwt}.

There are three primitives (one by thread state):

\begin{itemize}
\item \texttt{Lwt.return : 'a -> 'a Lwt.t} which create a thread which has already
  terminated and returned a value
\item \texttt{Lwt.fail : exn -> 'a Lwt.t} which create a thread which
  has already terminated and failed with an exception
\item \texttt{Lwt.wait : unit -> 'a Lwt.t * 'a Lwt.u} which create a
  sleeping thread and returns this thread plus a wakener (of type
  \texttt{'a Lwt.u}) which must be used to wakeup the sleeping thread.
\end{itemize}

To wakeup a sleeping thread, you can use one of the following
function:

\begin{itemize}
\item \texttt{Lwt.wakeup : 'a Lwt.u -> 'a -> unit} which wakeup the
  thread with a value.
\item \texttt{Lwt.wakeup\_exn : 'a Lwt.u -> exn -> unit} which wakeup
  the thread with an exception.
\end{itemize}

Not that this is an error to wakeup two times the same threads. \lwt
will raise \texttt{Invalid\_argument} if you try to do so.

With these informations, try to guess the result of each of the
following expression:

\begin{verbatim}
# Lwt.state (Lwt.return 42);;
# Lwt.state (Lwt.fail Exit);;
# let waiter, wakener = Lwt.wait ();;
# Lwt.state waiter;;
# Lwt.wakeup wakener 42;;
# Lwt.state waiter;;
# let waiter, wakener = Lwt.wait ();;
# Lwt.state waiter;;
# Lwt.wakeup_exn wakener Exit;;
# Lwt.state waiter;;
\end{verbatim}

\subsubsection{Primitives for thread composition}

The most important operation you need to know is \texttt{bind}:

\begin{verbatim}
val bind : 'a Lwt.t -> ('a -> 'b Lwt.t) -> 'b Lwt.t
\end{verbatim}

\texttt{bind t f} returns the thread which waits for \texttt{t} to
terminates, then pass the result to \texttt{f}. It \texttt{t} is a
sleeping thread, then \texttt{bind t f} will be a sleeping thread too,
until \texttt{t} terminates. If \texttt{t} fails, then the resulting
thread will fail with the same exception.

Similarly to \texttt{bind}, there is a function to handle the case
when \texttt{t} fails:

\begin{verbatim}
val catch : (unit -> 'a Lwt.t) -> (exn -> 'a Lwt.t) -> 'a Lwt.t
\end{verbatim}

\texttt{catch f g} will call \texttt{f ()}, then waits for its
termination, and if it fails with an exception \texttt{exn}, calls
\texttt{g exn} to handle it.

Look at the documentation of the \texttt{Lwt} module to see all
primitives.

\subsubsection{Cancelable threads}

In some case, we may want to cancel a threads. For example, because it
has not terminated after a timeout. This can be done with cancelable
threads. To create a cancelable threads, you must use the
\texttt{Lwt.task} function:

\begin{verbatim}
val task : unit -> 'a Lwt.t * 'a Lwt.u
\end{verbatim}

It has the same semantic that \texttt{Lwt.wait} except that the
sleeping thread can be canceled with \texttt{Lwt.cancel}:

\begin{verbatim}
val cancel : 'a Lwt.t -> unit
\end{verbatim}

The threads will then fails with the exception
\texttt{Lwt.Canceled}. To execute a function when the thread is
canceled, you must use \texttt{Lwt.on\_cancel}:

\begin{verbatim}
val on_cancel : 'a Lwt.t -> (unit -> unit) -> unit
\end{verbatim}

\subsubsection{Rules}

\lwt will always try to execute the more it can before yielding and
switch to another cooperative threads. In order to make it works well,
you must follow the following rules:

\begin{itemize}
\item do not write function that may takes time to complete
\item do not do IOs that may block, otherwise the whole program will
  hang. You must instead use asynchronous IOs operations
\end{itemize}

\subsection{The syntax extension}

\lwt offers a syntax extension to make code using it more readable.
To use it add the ``lwt.syntax'' when compiling:

\begin{verbatim}
$ ocamlfind ocamlc -syntax camlp4o -package lwt.syntax -linkpkg -o foo foo.ml
\end{verbatim}

Or in the toplevel (after loading topfind):

\begin{verbatim}
# #camlp4o;;
# #require "lwt.syntax";;
\end{verbatim}

The following construction are added to the language:

\begin{itemize}
\item \texttt{lwt} $pattern_1$ \texttt{=} $expr_1$ [ \texttt{and}
  $pattern_2$ \texttt{=} $expr_2$ \dots ] \texttt{in} $expr$, which is a
  parallel let-binding construction. For example in the following
  code:

\begin{verbatim}
lwt x = f () and y = g () in
expr
\end{verbatim}

  the thread \texttt{f ()} and \texttt{g ()} are launched in parallel
  and their result are then bound to \texttt{x} and \texttt{y} in the
  expression $expr$.

  Of course you can also launch the two threads sequentially by
  writing your code like that:

\begin{verbatim}
lwt x = f () in
lwt y = g () in
expr
\end{verbatim}

\item \texttt{try\_lwt} $expr$ [ \texttt{with} $pattern_1$
  \texttt{$\rightarrow$} $expr_1$ ... ] [ \texttt{finally} $expr'$ ],
  which is the equivalent of the standard \texttt{try-with}
  construction but for \lwt. Both exception raised by
  \texttt{Pervasives.raise} and \texttt{Lwt.fail} are caught.

\item \texttt{for\_lwt} $ident$ \texttt{=} $expr_{init}$ ( \texttt{to}
  $\mid$ \texttt{downto} ) $expr_{final}$ \texttt{do} $expr$
  \texttt{done}, which is the equivalent of the standard \texttt{for}
  construction but for \lwt.
\end{itemize}

\subsubsection{Correspondence table}

You can keep in mind the following table to write code using lwt:

\begin{center}
  \begin{tabular}{|l|l|}
    \hline

    \textbf{without \lwt} & \textbf{with \lwt} \\

    \hline

    \texttt{let} $pattern_1$ \texttt{=} $expr_1$ &
    \texttt{lwt} $pattern_1$ \texttt{=} $expr_1$ \\
    \texttt{and} $pattern_2$ \texttt{=} $expr_2$ &
    \texttt{and} $pattern_2$ \texttt{=} $expr_2$ \\
    \dots &
    \dots \\
    \texttt{and} $pattern_n$ \texttt{=} $expr_n$ \texttt{in} &
    \texttt{and} $pattern_n$ \texttt{=} $expr_n$ \texttt{in} \\
    $expr$ &
    $expr$ \\

    \hline

    \texttt{try} &
    \texttt{try\_lwt} \\
    \:\: $expr$ &
    \:\: $expr$ \\
    \texttt{with} &
    \texttt{with} \\
    \:\: $\mid pattern_1$ \texttt{$\rightarrow$} $expr_1$ &
    \:\: $\mid pattern_1$ \texttt{$\rightarrow$} $expr_1$ \\
    \:\: $\mid pattern_2$ \texttt{$\rightarrow$} $expr_2$ &
    \:\: $\mid pattern_2$ \texttt{$\rightarrow$} $expr_2$ \\
    \:\: \dots &
    \:\: \dots \\
    \:\: $\mid$ $pattern_n$ \texttt{$\rightarrow$} $expr_n$ &
    \:\: $\mid$ $pattern_n$ \texttt{$\rightarrow$} $expr_n$ \\

    \hline

    \texttt{for} $ident$ \texttt{=} $expr_{init}$ \texttt{to} $expr_{final}$ \texttt{do} &
    \texttt{for\_lwt} $ident$ \texttt{=} $expr_{init}$ \texttt{to} $expr_{final}$ \texttt{do} \\
    \:\: $expr$ &
    \:\: $expr$ \\
    \texttt{done} &
    \texttt{done} \\

    \hline

    \texttt{for} $ident$ \texttt{=} $expr_{init}$ \texttt{downto} $expr_{final}$ \texttt{do} &
    \texttt{for\_lwt} $ident$ \texttt{=} $expr_{init}$ \texttt{downto} $expr_{final}$ \texttt{do} \\
    \:\: $expr$ &
    \:\: $expr$ \\
    \texttt{done} &
    \texttt{done} \\

    \hline
  \end{tabular}
\end{center}

\subsection{Other modules of the core library}

The core library contains several modules that depends only on
\lwt. Amongst them are:

\begin{itemize}
\item \texttt{Lwt\_list}: it defines iteration and scanning functions
  over list using functions that returns a thread. For example:

\begin{verbatim}
val map_s : ('a -> 'b Lwt.t) -> 'a list -> 'b list Lwt.t
val map_p : ('a -> 'b Lwt.t) -> 'a list -> 'b list Lwt.t
\end{verbatim}

  Note that the following naming convention is used: when the function
  taken as argument is going to be executed sequentially, the
  ``\texttt{\_s}'' suffix is used, and when it is going to be executed
  in parallel, the suffix ``\texttt{\_p}'' is used.

\item \texttt{Lwt\_mutex}: it defines mutex for \lwt. Basically the
  only goal of an \lwt mutex is to serialize operations.

\item \texttt{Lwt\_stream}: a stream is a object which returns
  elements sequentially and lazily. Lazily means that the source of
  the stream is guessed for new elements only when needed. This module
  contains a lot of stream transformation, iteration, and scanning
  functions.

  Note that streams are consumable, if you want to iterates two times
  over a stream, you may consider ``clonning'' it (via
  \texttt{Lwt\_stream.clone}).
\end{itemize}

%% +-----------------------------------------------------------------+
%% | Section                                                         |
%% +-----------------------------------------------------------------+
\section{The lwt.unix library}
\label{lwt.unix}

The package \texttt{lwt.unix} contains all \texttt{unix} dependant
modules of \lwt. It provides cooperative versions of functions of the
standard and the unix libraries.

The \texttt{Lwt\_unix} module is the cooperative equivalent of the
\texttt{Unix} module, plus some specific functions. It also provides
an helper to wrap system calls into cooperatives one (the
\texttt{wrap\_syscall} function).

\texttt{Lwt\_io} provides similar features that buffered channels of
the standard library (of type \texttt{in\_channel} or
\texttt{out\_channel}) but cooperatively.

\texttt{Lwt\_gc} allow you to register finaliser that return a
thread. At the end of the program, \lwt will wait for all the
finaliser to terminates.

\subsection{The lwt scheduler}

The \texttt{Lwt\_main} contains the \lwt \emph{main loop}. It can be
customized by adding filters, and/or by replacing the \texttt{select}
function.

Filters are responsible to collect sources to monitor before entering
the blocking \texttt{select}, then to react and wakeup threads waiting
for sources to become ready.

\subsection{Limitations}

To avoid blocking, \lwt puts all file descriptors (except
\texttt{stdin}, \texttt{stdout} and \texttt{stderr}) in non-blocking
mode. Unfortunatly no unix supports non-blocking mode for regular
files. In practise a file descriptor for a regular file will always
appears to be readable and writable even if this is not the case (for
example because the file is located on a network file system and the
server is not available). There is no workaround for this.

By the way, since reading/writing a file will never appears to block,
reading or writing a big file may prevent other threads to run until
the operation on the file are finished. To avoid this,
\texttt{Lwt\_io} yield every 0.05 seconds for regular files.

Since there is no support at all for asynchrnously reading the
contents of a directory, \texttt{Lwt\_directory} uses the same trick.

%% +-----------------------------------------------------------------+
%% | Section                                                         |
%% +-----------------------------------------------------------------+
\section{Detaching computation to preemptive threads}

It may happen that you want to run a function which will take time to
compute or that you want to use a blocking function that cannot be
used in a non-blocking way. For these situations, \lwt allow you to
\emph{detach} the computation to a preemptive thread.

This is done by the module \texttt{Lwt\_preemptive} of the
\texttt{lwt.preemptive} package which maintains a spool of system
threads. The main function is:

\begin{verbatim}
val detach : ('a -> 'b) -> 'a -> 'b Lwt.t
\end{verbatim}

\texttt{detach f x} will execute \texttt{f x} in another thread and
asynchronously wait for the result.

The \texttt{lwt.extra} package provides wrappers for a few blocking
functions of the standard C library like \texttt{gethostbyname} (in
the module \texttt{Lwt\_lib}).

%% +-----------------------------------------------------------------+
%% | Section                                                         |
%% +-----------------------------------------------------------------+
\section{The lwt.text library}

The \texttt{lwt.text} library provides functions to deal with text
mode (in a terminal). It is composed of three following modules:

\begin{itemize}
\item \texttt{Lwt\_text}, which is the equivalent of \texttt{Lwt\_io}
  but for unicode text channels
\item \texttt{Lwt\_term}, providing various terminal utilities, such as
  reading a key from the terminal
\item \texttt{Lwt\_read\_line}, which provides functions to input text
  from the user with line editing support
\end{itemize}

\subsection{Text channels}

A text channel is basically a byte channel plus an encoding. Input
(resp. output) text channels decode (resp. encode) unicode characters
on the fly. By default, output text channels use transliteration, so
they will not fails because text you want to print cannot be encoded
in the system encoding.

For example, with you locale sets to ``C'', and the variable
\texttt{name} set to ``Jérémie'', you got:

\begin{verbatim}
# lwt () = Lwt_text.printlf "My name is %s" name;;
My name is J?r?mie
\end{verbatim}

\subsection{Terminal utilities}

The \texttt{Lwt\_term} allow you to put the terminal in \emph{raw
  mode}, meanings that input is not buffered and character are
returned as the user type them. For example, you can read a key with:

\begin{verbatim}
# lwt key = Lwt_term.read_key ();;
val key : Lwt_term.key = Lwt_term.Key_control 'j'
\end{verbatim}

The second main feature of \texttt{Lwt\_term} is the ability to prints
text with styles. For example, to print text in bold and blue:

\begin{verbatim}
# open Lwt_term;;
# lwt () = printlc [fg blue; bold; text "foo"];;
foo
\end{verbatim}

If the output is not a terminal, then \texttt{printlc} will drop
styles, and act as \texttt{Lwt\_text.printl}.

\subsection{Read-line}

\texttt{Lwt\_read\_line} provides a full featured and fully
customisable read-line implementation. You can either use the
high-level and easy to use \texttt{read\_*} functions, or use the
advanced \texttt{Lwt\_read\_line.Control.read\_*} functions.

For example:

\begin{verbatim}
# open Lwt_term;;
# lwt l = Lwt_read_line.read_line ~prompt:[text "foo> "] ();;
foo> Hello, world!
val l : Text.t = "Hello, world!"
\end{verbatim}

The second class of functions is a bit more complicated to use, but
allow to control a running read-line instances. For example you can
temporary hide it to draw something, you can send it commands, fake
input, and the prompt is a signal so it can change dynamically.

%% +-----------------------------------------------------------------+
%% | Section                                                         |
%% +-----------------------------------------------------------------+
\section{The logging facility}
\label{lwt.log}

The package \texttt{lwt.unix} contains a module \texttt{Lwt\_log}
providing loggers. It support logging to a file, a channel, or to the
syslog daemon. You can also defines your own logger by providing the
appropriate functions (function \texttt{Lwt\_log.make}).

Several loggers can be merged into one. Sending logs on the merged
loggers will send these logs to all its components.

For example to redirect all logs to \texttt{stderr} and to the syslog
daemon:

\begin{verbatim}
# Lwt_log.default_logger :=
    Lwt_log.merge ~level:Lwt_log.Info [
      Lwt_log.channel
        ~level:Lwt_log.Info
        ~close_mode:`keep
        ~channel:Lwt_io.stderr ();
      Lwt_log.syslog
        ~level:Lwt_log.Info
        ~facility:`User ();
    ]
;;
\end{verbatim}

\lwt also provides a syntax extension, in the package
\texttt{lwt.syntax.log}. It adds the following construction to the
language:

\begin{center}
  \texttt{Log\#}\textit{level} \textit{format} \textit{arguments}
\end{center}

Where \textit{level} is one of \texttt{debug}, \texttt{info},
\texttt{warning}, \texttt{error}, \texttt{fatal}. For example:

\begin{verbatim}
Log#debug "x = %s" x
\end{verbatim}

The advantages of using the syntax extension are the following:

\begin{itemize}
\item it automatically prefix the format with the module name
\item it check the log level before calling the logging function, so
  arguments are not computed if not needed
\item debugging logs can be removed at parsing time
\end{itemize}

By default, the syntax extension remove all logs with the level
\texttt{debug}. To keep them pass the command line option
\texttt{-lwt-debug} to camlp4.

%% +-----------------------------------------------------------------+
%% | Section                                                         |
%% +-----------------------------------------------------------------+
\section{Other libraries}

\subsection{SSL support}

The package \texttt{lwt.ssl} provides the module \texttt{Lwt\_ssl}
which allow to use SSL asynchronously

\subsection{Glib integration}

The \texttt{lwt.glib} provides an alternative main loop for \lwt which
uses the glib one. This allow you to write GTK application using \lwt.
The one thing you have to do is to call \texttt{Lwt\_glib.init} at the
beginning of you program.

\end{document}
