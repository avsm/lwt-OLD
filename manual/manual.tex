\documentclass{article}

\usepackage[utf8]{inputenc}
\usepackage{xspace}
\usepackage{verbatim}
\usepackage[english]{babel}
\usepackage{amsmath}
\usepackage{amssymb}

%% +-----------------------------------------------------------------+
%% | Aliases                                                         |
%% +-----------------------------------------------------------------+

\newcommand{\lwt}{\texttt{Lwt}\xspace}

%% +-----------------------------------------------------------------+
%% | Headers                                                         |
%% +-----------------------------------------------------------------+

\title{Lwt user manual}
\author{Jérémie Dimino}

\begin{document}

\maketitle

%% +-----------------------------------------------------------------+
%% | Table of contents                                               |
%% +-----------------------------------------------------------------+

\setcounter{tocdepth}{2}
\tableofcontents

%% +-----------------------------------------------------------------+
%% | Section                                                         |
%% +-----------------------------------------------------------------+
\section{Introduction}

%% +-----------------------------------------------------------------+
%% | Section                                                         |
%% +-----------------------------------------------------------------+
\section{The Lwt core library}

\subsection{The Lwt syntax extension}

\lwt offers a syntax extension to make code using it more readable.
The following construction are added to the language:

\begin{itemize}
\item \texttt{lwt} $pattern_1$ \texttt{=} $expr_1$ [ \texttt{and}
  $pattern_2$ \texttt{=} $expr_2$ \dots ] \texttt{in} $expr$, which is a
  parallel let-binding construction. For example in the following
  code:

\begin{verbatim}
lwt x = f () and y = g () in
expr
\end{verbatim}

  the thread \texttt{f ()} and \texttt{g ()} are launched in parallel
  and their result are then bound to \texttt{x} and \texttt{y} in the
  expression $expr$.

  Of course you can also launch the two threads sequentially by
  writing your code like that:

\begin{verbatim}
lwt x = f () in
lwt y = g () in
expr
\end{verbatim}

\item \texttt{try\_lwt} $expr$ [ \texttt{with} $pattern_1$
  \texttt{$\rightarrow$} $expr_1$ ... ] [ \texttt{finally} $expr'$ ],
  which is the equivalent of the standard \texttt{try-with}
  construction but for \lwt. Both exception raised by
  \texttt{Pervasives.raise} and \texttt{Lwt.fail} are caught.

\item \texttt{for\_lwt} $ident$ \texttt{=} $expr_{init}$ ( \texttt{to}
  $\mid$ \texttt{downto} ) $expr_{final}$ \texttt{do} $expr$
  \texttt{done}, which is the equivalent of the standard \texttt{for}
  construction but for \lwt.
\end{itemize}

\subsubsection{Correspondence table}

You can keep in mind the following table to write code using lwt:

\begin{center}
  \begin{tabular}{|l|l|}
    \hline

    \textbf{without \lwt} & \textbf{with \lwt} \\

    \hline

    \texttt{let} $pattern_1$ \texttt{=} $expr_1$ &
    \texttt{lwt} $pattern_1$ \texttt{=} $expr_1$ \\
    \texttt{and} $pattern_2$ \texttt{=} $expr_2$ &
    \texttt{and} $pattern_2$ \texttt{=} $expr_2$ \\
    \dots &
    \dots \\
    \texttt{and} $pattern_n$ \texttt{=} $expr_n$ \texttt{in} &
    \texttt{and} $pattern_n$ \texttt{=} $expr_n$ \texttt{in} \\
    $expr$ &
    $expr$ \\

    \hline

    \texttt{try} &
    \texttt{try\_lwt} \\
    \:\: $expr$ &
    \:\: $expr$ \\
    \texttt{with} &
    \texttt{with} \\
    \:\: $\mid pattern_1$ \texttt{$\rightarrow$} $expr_1$ &
    \:\: $\mid pattern_1$ \texttt{$\rightarrow$} $expr_1$ \\
    \:\: $\mid pattern_2$ \texttt{$\rightarrow$} $expr_2$ &
    \:\: $\mid pattern_2$ \texttt{$\rightarrow$} $expr_2$ \\
    \:\: \dots &
    \:\: \dots \\
    \:\: $\mid$ $pattern_n$ \texttt{$\rightarrow$} $expr_n$ &
    \:\: $\mid$ $pattern_n$ \texttt{$\rightarrow$} $expr_n$ \\

    \hline

    \texttt{for} $ident$ \texttt{=} $expr_{init}$ \texttt{to} $expr_{final}$ \texttt{do} &
    \texttt{for\_lwt} $ident$ \texttt{=} $expr_{init}$ \texttt{to} $expr_{final}$ \texttt{do} \\
    \:\: $expr$ &
    \:\: $expr$ \\
    \texttt{done} &
    \texttt{done} \\

    \hline

    \texttt{for} $ident$ \texttt{=} $expr_{init}$ \texttt{downto} $expr_{final}$ \texttt{do} &
    \texttt{for\_lwt} $ident$ \texttt{=} $expr_{init}$ \texttt{downto} $expr_{final}$ \texttt{do} \\
    \:\: $expr$ &
    \:\: $expr$ \\
    \texttt{done} &
    \texttt{done} \\

    \hline
  \end{tabular}
\end{center}

%% +-----------------------------------------------------------------+
%% | Section                                                         |
%% +-----------------------------------------------------------------+
\section{The Lwt.unix library}

%% +-----------------------------------------------------------------+
%% | Section                                                         |
%% +-----------------------------------------------------------------+
\section{Detaching computation to preemptive threads}

\subsection{Lwt.preemptive}
\subsection{Lwt.extra}

%% +-----------------------------------------------------------------+
%% | Section                                                         |
%% +-----------------------------------------------------------------+
\section{The Lwt.text library}

%% +-----------------------------------------------------------------+
%% | Section                                                         |
%% +-----------------------------------------------------------------+
\section{The logging facility}

%% +-----------------------------------------------------------------+
%% | Section                                                         |
%% +-----------------------------------------------------------------+
\section{Other libraries}

\subsection{Glib integration}
\subsection{SSL support}

\end{document}
